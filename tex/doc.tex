
\documentclass[12pt]{article}
\usepackage[english]{babel}
\usepackage{amsfonts}
\usepackage{amsmath}
\usepackage{amssymb}
\usepackage{mathdots}
\usepackage{hyperref}
\usepackage{graphicx}
\usepackage{url}
\usepackage[T1]{fontenc}
\usepackage{euler} % for math
\usepackage{beramono} % for typewriter
\usepackage{newtxtext} % for text
\usepackage{minted}

\textwidth 16cm \textheight 23cm \topmargin -1cm \oddsidemargin 0cm


%%%%%%%%%%%%%%%%%%%%%%%%%%%%%%%%%%%%%%%%%%%%%%%%%%%%%%%%%%%%%%%%%%%%%

\begin{document}


\title{\texttt{unittest} and \texttt{unittest.mock}\\ Python modules: some exercises}
\author{Massimo Nocentini\\
Dipartimento di Statistica, Informatica, Applicazioni \\ 
viale Morgagni 65, 50134, Firenze, Italia \\ 
{\sl  massimo.nocentini@unifi.it}}

\date{\today}

\maketitle

\begin{abstract}

This note collects three exercises about \textit{Test-Driven Development}
and \textit{Mocking} methodologies; in particular, \texttt{unittest} and
\texttt{unittest.mock} Python modules are investigated and used
in order to simulate the Java counterpart given in \cite{course}. 
 
\end{abstract}

\section{Introduction}

Our aim is to deepen the understanding of two Python modules, namely
\texttt{unittest} and \texttt{unittest.mock}, and use them in an \emph{agile
programming} context: the former concerns \textit{TDD}, the latter concerns
\textit{mocking}.  Both modules belong to the Python Standard Library
\cite{psl} and are used in two toy projects, \texttt{factorial} and
\texttt{payroll} respectively, to drive development in order to write clean
and maintainable pieces of code; moreover, we bootstrap a tiny testing framework
from scratch. The following sections describe them in more details.

\section{Toy projects}

\subsection{\texttt{factorial}}

\inputminted{python}{../factorial/factorial.py}
\inputminted{python}{../factorial/factorial_test.py}

\subsection{\texttt{payroll}}

\subsection{\texttt{bootstrapping}}

\section{GitHub and Travis CI}

\section{Conclusions}

% Bibliography {{{

\begin{thebibliography}{10}

\bibitem{course}
Lorenzo Bettini,
\newblock {\em B026351 (B059) - Tecniche Avanzate di Programmazione, 2016/2017}
\newblock \url{https://e-l.unifi.it/course/view.php?id=2215}
%
\bibitem{psl}
Python Software Foundation,
\newblock {\em Python Standard Library},
\newblock \url{https://docs.python.org/3/library/}
%
\bibitem{beck}
Kent Beck,
\newblock {\em Test Driven Development: By Example},
\newblock Addison-Wesley, 2002
%
%\bibitem{FKT88}
%P.~Flajolet, P.~Kirschenhofer, and R.~F. Tichy.
%\newblock Deviations from uniformity in random strings.
%\newblock {\em Probability Theory and Related Fields}, 80:139--150, 1988.
%
%\bibitem{FS09}
%P.~Flajolet and R.~Sedgewick.
%\newblock {\em Analytic combinatorics}.
%\newblock Cambridge University Press, 2009.
%
%\bibitem{FSV06}
%P.~Flajolet, W.~Szpankowski, and B.~Vall\'ee.
%\newblock Hidden word statistics.
%\newblock {\em Journal of the ACM}, 53(1):147--183, 2006.
%
%\bibitem{GO80}
%L.~J. Guibas and M.~Odlyzko.
%\newblock Long repetitive patterns in random sequences.
%\newblock {\em Zeitschrift f\"{u}r Wahrscheinlichkeitstheorie}, 53:241--262,
%  1980.
%
%\bibitem{GO81}
%L.~J. Guibas and M.~Odlyzko.
%\newblock String overlaps, pattern matching, and nontransitive games.
%\newblock {\em Journal of Combinatorial Theory, Series A}, 30:183--208, 1981.
%
%\bibitem{KMP77}
%D.~E. Knuth, J.~H. Morris, and V.~R. Pratt.
%\newblock Fast pattern matching in strings.
%\newblock {\em SIAM journal on Computing}, 6:281--315, 1977.
%
%\bibitem{KS94}
%S.~Kumar and E.~H. Spafford.
%\newblock A pattern matching model for misuse intrusion detection.
%\newblock In {\em Proceedings of the 17th National Computer Security
%  Conference}, pages 11--21, 1994.
%
%\bibitem{Mer95}
%D.~Merlini.
%\newblock I {R}iordan {A}rray nell'{An}alisi degli {A}lgoritmi.
%\newblock Tesi di Dottorato, Universit{\`a} degli Studi di Firenze, 1996.
%
%\bibitem{MRSV97}
%D.~Merlini, D.~G. Rogers, R.~Sprugnoli, and M.~C. Verri.
%\newblock On some alternative characterizations of {R}iordan arrays.
%\newblock {\em Canadian Journal of Mathematics}, 49(2):301--320, 1997.
%
%\bibitem{RFPGP00}
%I.~Rigoutsos, A.~Floratos, L.~Parida, Y.~Gao, and D.~Platt.
%\newblock The emergence of pattern discovery techniques in computational
%  biology.
%\newblock {\em Metabolic Engineering}, 2:159--177, 2000.
%
%\bibitem{SF96}
%R.~Sedgewick and P.~Flajolet.
%\newblock {\em An {I}ntroduction to the {A}nalysis of {A}lgorithms}.
%\newblock Addison-Wesley, Reading, MA, 1996.
%
%\bibitem{SGWW91}
%L.~W. Shapiro, S.~Getu, W.-J. Woan, and L.~Woodson.
%\newblock The {R}iordan group.
%\newblock {\em Discrete Applied Mathematics}, 34:229--239, 1991.
%
%\bibitem{Spr94}
%R.~Sprugnoli.
%\newblock Riordan arrays and combinatorial sums.
%\newblock {\em Discrete Mathematics}, 132:267--290, 1994.
%
%\bibitem{Sta99}
%R.~P. Stanley.
%\newblock {\em Enumerative {C}ombinatorics}, volume~2.
%\newblock Cambridge University Press, Cambridge, 1999.
%
%\bibitem{Wat95}
%M.~Waterman.
%\newblock {\em Introduction to Computational Biology}.
%\newblock Chapman \& Hall, London, 1995.
%
\end{thebibliography}
% }}}

\end{document}

