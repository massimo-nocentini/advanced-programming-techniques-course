
\documentclass[12pt]{article}
\usepackage[english]{babel}
\usepackage{amsfonts}
\usepackage{amsmath}
\usepackage{amssymb}
\usepackage{mathdots}
\usepackage{hyperref}
\usepackage{graphicx}
\usepackage{url}
\usepackage[T1]{fontenc}
\usepackage{euler} % for math
\usepackage{beramono} % for typewriter
\usepackage{newtxtext} % for text

\textwidth 16cm \textheight 23cm \topmargin -1cm \oddsidemargin 0cm


%%%%%%%%%%%%%%%%%%%%%%%%%%%%%%%%%%%%%%%%%%%%%%%%%%%%%%%%%%%%%%%%%%%%%

\begin{document}


\title{Algebraic generating functions for languages \\ avoiding Riordan patterns}
\author{D. Merlini and M. Nocentini\\
Dipartimento di Statistica, Informatica, Applicazioni \\ 
viale Morgagni 65, 50134, Firenze, Italia \\ 
{\sl donatella.merlini@unifi.it, massimo.nocentini@unifi.it}}
\date{\today}
\maketitle
\begin{abstract}

We study the languages  $\mathfrak{L}^{[\mathfrak{p}]}\subset \{0,1\}^*$ of
binary words $w$ avoiding a given pattern $\mathfrak{p}$ such that $|w_0|\leq
|w_1|$ for any $w\in \mathfrak{L}^{[\mathfrak{p}]},$ where  $|w_0|$ and $|w_1|$
correspond to the number of bits $1$ and $0$ in the word $w$, respectively.  In
particular, we concentrates on  patterns $\mathfrak{p}$ related to the concept
of Riordan arrays. These languages are not regular and can be enumerated by
algebraic generating functions corresponding to many integer sequences which
are unknown in the OEIS .  We give explicit formulas for these generating
functions expressed in terms of the autocorrelation polynomial of
$\mathfrak{p}$ and also give explicit formulas for the coefficients of some
particular patterns, algebraically and combinatorially. 

\end{abstract}

\section{Introduction}


% Bibliography {{{
%%%%%%%%%%%%%%%%%%%%%%%%%%%%%%%%%%%%%%%%%%%%%%%%%%%%%%%%%%%%%%%%%%%%%
%\bibliographystyle{plain}
%\bibliography{../../tex/bibliogr}
%%%%%%%%%%%%%%%%%%%%%%%%%%%%%%%%%%%%%%%%%%%%%%%%%%%%%%%%%%%%%%%%%%%%%

%\begin{thebibliography}{10}
%
%\bibitem{AA02}
%A.~Apostolico and M.~Atallah.
%\newblock Compact recognizers of episode sequences.
%\newblock {\em Information and Computation}, 174(2):180--192, 2002.
%
%\bibitem{BMS07a}
%D.~Baccherini, D.~Merlini, and R.~Sprugnoli.
%\newblock Binary words excluding a pattern and proper {R}iordan arrays.
%\newblock {\em Discrete Mathematics}, 307:1021--1037, 2007.
%
%\bibitem{CS63}
%N.~Chomsky and M.~P. Sch\"utzenberger.
%\newblock The algebraic theory of context-free languages.
%\newblock In {\em Computer programming and formal languages}, volume~65, pages
%  118--161, North Holland, 1963.
%
%\bibitem{FKT88}
%P.~Flajolet, P.~Kirschenhofer, and R.~F. Tichy.
%\newblock Deviations from uniformity in random strings.
%\newblock {\em Probability Theory and Related Fields}, 80:139--150, 1988.
%
%\bibitem{FS09}
%P.~Flajolet and R.~Sedgewick.
%\newblock {\em Analytic combinatorics}.
%\newblock Cambridge University Press, 2009.
%
%\bibitem{FSV06}
%P.~Flajolet, W.~Szpankowski, and B.~Vall\'ee.
%\newblock Hidden word statistics.
%\newblock {\em Journal of the ACM}, 53(1):147--183, 2006.
%
%\bibitem{GO80}
%L.~J. Guibas and M.~Odlyzko.
%\newblock Long repetitive patterns in random sequences.
%\newblock {\em Zeitschrift f\"{u}r Wahrscheinlichkeitstheorie}, 53:241--262,
%  1980.
%
%\bibitem{GO81}
%L.~J. Guibas and M.~Odlyzko.
%\newblock String overlaps, pattern matching, and nontransitive games.
%\newblock {\em Journal of Combinatorial Theory, Series A}, 30:183--208, 1981.
%
%\bibitem{KMP77}
%D.~E. Knuth, J.~H. Morris, and V.~R. Pratt.
%\newblock Fast pattern matching in strings.
%\newblock {\em SIAM journal on Computing}, 6:281--315, 1977.
%
%\bibitem{KS94}
%S.~Kumar and E.~H. Spafford.
%\newblock A pattern matching model for misuse intrusion detection.
%\newblock In {\em Proceedings of the 17th National Computer Security
%  Conference}, pages 11--21, 1994.
%
%\bibitem{Mer95}
%D.~Merlini.
%\newblock I {R}iordan {A}rray nell'{An}alisi degli {A}lgoritmi.
%\newblock Tesi di Dottorato, Universit{\`a} degli Studi di Firenze, 1996.
%
%\bibitem{MRSV97}
%D.~Merlini, D.~G. Rogers, R.~Sprugnoli, and M.~C. Verri.
%\newblock On some alternative characterizations of {R}iordan arrays.
%\newblock {\em Canadian Journal of Mathematics}, 49(2):301--320, 1997.
%
%\bibitem{RFPGP00}
%I.~Rigoutsos, A.~Floratos, L.~Parida, Y.~Gao, and D.~Platt.
%\newblock The emergence of pattern discovery techniques in computational
%  biology.
%\newblock {\em Metabolic Engineering}, 2:159--177, 2000.
%
%\bibitem{SF96}
%R.~Sedgewick and P.~Flajolet.
%\newblock {\em An {I}ntroduction to the {A}nalysis of {A}lgorithms}.
%\newblock Addison-Wesley, Reading, MA, 1996.
%
%\bibitem{SGWW91}
%L.~W. Shapiro, S.~Getu, W.-J. Woan, and L.~Woodson.
%\newblock The {R}iordan group.
%\newblock {\em Discrete Applied Mathematics}, 34:229--239, 1991.
%
%\bibitem{Spr94}
%R.~Sprugnoli.
%\newblock Riordan arrays and combinatorial sums.
%\newblock {\em Discrete Mathematics}, 132:267--290, 1994.
%
%\bibitem{Sta99}
%R.~P. Stanley.
%\newblock {\em Enumerative {C}ombinatorics}, volume~2.
%\newblock Cambridge University Press, Cambridge, 1999.
%
%\bibitem{Wat95}
%M.~Waterman.
%\newblock {\em Introduction to Computational Biology}.
%\newblock Chapman \& Hall, London, 1995.
%
%\end{thebibliography}
% }}}

\end{document}

