
\documentclass[12pt]{article}
\usepackage[english]{babel}
\usepackage{amsfonts}
\usepackage{amsmath}
\usepackage{amssymb}
\usepackage{mathdots}
\usepackage{hyperref}
\usepackage{graphicx}
\usepackage{url}
\usepackage[T1]{fontenc}
\usepackage{euler} % for math
\usepackage{beramono} % for typewriter
\usepackage{newtxtext} % for text
\usepackage{minted}

\textwidth 16cm \textheight 23cm \topmargin -1cm \oddsidemargin 0cm


%%%%%%%%%%%%%%%%%%%%%%%%%%%%%%%%%%%%%%%%%%%%%%%%%%%%%%%%%%%%%%%%%%%%%

\begin{document}


\title{\texttt{unittest} and \texttt{unittest.mock}\\ Python modules: some exercises}
\author{Massimo Nocentini\\
Dipartimento di Statistica, Informatica, Applicazioni \\ 
viale Morgagni 65, 50134, Firenze, Italia \\ 
{\sl  massimo.nocentini@unifi.it}}

\date{\today}

\maketitle

\begin{abstract}

This note collects three exercises about \textit{Test-Driven Development}
and \textit{Mocking} methodologies; in particular, \texttt{unittest} and
\texttt{unittest.mock} Python modules are investigated and used
in order to simulate the Java counterpart given in \cite{course}. 
 
\end{abstract}

\section{Introduction}

Our aim is to deepen the understanding of two Python modules, namely
\texttt{unittest} and \texttt{unittest.mock}, and use them in an \emph{agile
programming} context: the former concerns \textit{TDD}, the latter concerns
\textit{mocking}, as their names suggest.  Both modules belong to the Python
Standard Library \cite{psl}, available for both Python 2.7 and 3.x releases;
eventually, we code two toy projects, \texttt{factorial} and \texttt{payroll}
respectively, using the provided definitions in order to write clean and
maintainable pieces of code.  As a side project, we bootstrap a tiny testing
framework from scratch, all versioned and integrated on top of \emph{GitHub}
and \emph{Travis CI}. The following sections describe them in more details.

\section{Toy projects}

\subsection{\texttt{factorial}}
In this exercise we define and implement the factorial function $n!$ for $n\in\mathbb{N}$,
driven by test cases using the \texttt{unittest} module \cite{unittest}; 
the following definitions read as a \emph{runnable specification}:
\inputminted{python}{../factorial/factorial_test.py}
Therefore a possible implementation satisfying the given requirements reads as follows:
\inputminted{python}{../factorial/factorial.py}
\noindent where it is interesting to observe that implementation uses iteration, while the specification
is given by recursion. Moreover, examples seen during classes have been easily ported on top
of \texttt{unittest} module.

\subsection{\texttt{payroll}}

In this exercise we want to \emph{mock} dependencies of
\mintinline{python}{Payroll} objects to test them in isolation using
definitions provided by the \texttt{unittest.mock} module \cite{unittest.mock}.
Here is the system under test: \inputminted[firstline=1,
lastline=19]{python}{../payroll/payroll.py} 
\noindent where dependecies are given in the
ctor and interaction happens in method \mintinline{python}{monthly_payment}.

On the other hand, in the corresponding test class, dependencies are mocked
with \texttt{unittest.mock.Mock} objects in the \texttt{setUp} method and an
example of \emph{stubbing} is given in the following chunk:
\inputminted[firstline=20, lastline=30]{python}{../payroll/payroll_test.py}
Moreover, the following methods have been implemented to show a parallel with
objects and methods provided by \texttt{Mockito}, a Java mocking framework for
what concerns interception of a single method call: \inputminted[firstline=42,
lastline=51]{python}{../payroll/payroll_test.py} and about a sequence of method
calls: \inputminted[firstline=63,
lastline=73]{python}{../payroll/payroll_test.py} Finally, all the code shown in
Java has been ported in test class \texttt{PayrollTest}.

\subsection{\texttt{bootstrapping}}
In this exercise we experience the bootstrap of a unit test framework, 
driven by test itself; in parallel, we find interesting to look at version
history to understand the steps that make it happen, strictly following
\cite{beck}. The driver reads as follows:
\inputminted{python}{../tdd/tdd_test.py}

\section{GitHub and Travis CI}

All code has been versioned using \emph{Git} and hosted at \cite{repo}; moreover,
a link with \emph{Travis CI} server \cite{travis} has been established in order to have
automatic builds for both branches and pull requests, targeting two 
different releases of Python.
%\inputminted{yaml}{../.travis.yml}

\section{Conclusions}

We have shown two ports of simple exercises using the Python language,
in order to understand and translate TDD and mock concept provided by
Java framework in pure Python; additionally, a simple but working
test framework has been implemented from scratch.

% Bibliography {{{

\begin{thebibliography}{10}

\bibitem{course}
Lorenzo Bettini,
\newblock {\em B026351 (B059) - Tecniche Avanzate di Programmazione, 2016/2017}
\newblock \url{https://e-l.unifi.it/course/view.php?id=2215}
%
\bibitem{psl}
Python Software Foundation,
\newblock {\em Python Standard Library},
\newblock \url{https://docs.python.org/3/library/}
%
\bibitem{beck}
Kent Beck,
\newblock {\em Test Driven Development: By Example},
\newblock Addison-Wesley, 2002
%
\bibitem{unittest}
Python Software Foundation,
\newblock {\em \texttt{unittest} -- Unit testing framework},
\newblock \url{https://docs.python.org/3/library/unittest.html}
%
\bibitem{unittest.mock}
Python Software Foundation,
\newblock {\em \texttt{unittest.mock} -- mock object library},
\newblock \url{https://docs.python.org/3/library/unittest.mock.html}
%
\bibitem{repo}
Massimo Nocentini,
\newblock \url{https://github.com/massimo-nocentini/advanced-programming-techniques-course}
%
\bibitem{travis}
Massimo Nocentini,
\newblock \url{https://travis-ci.org/massimo-nocentini/advanced-programming-techniques-course}
%
\end{thebibliography}
% }}}

\end{document}

